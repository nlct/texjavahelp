% arara: pdflatex
% arara: bib2gls: { group: on }
% arara: pdflatex
% arara: pdflatex
\documentclass[toc=listof]{scrreport}

\usepackage{texjavahelp}

\hypersetup{colorlinks,linkcolor=blue}

\title{\texorpdfstring{\TeX}{TeX} Java Help}
\author{Nicola L.C. Talbot}

\GlsXtrLoadResources[\TeXJavaHelpGlsResourceOptions]

\begin{document}
\maketitle
\tableofcontents

\chapter{Introduction}
\label{sec:intro}

This is the documentation for \TeX\ Java Help.

\begin{warning}
Under construction.
\end{warning}

\begin{itemize}
\item \sty{texjavahelp}: a \LaTeX\ package to help create the \gls+{pdf} version of
the documentation from the \LaTeX\ source.

\item \app{texjavahelpmk}: a \gls{cli} application that uses the \TeX\ Parser
Library to create the \gls+{html} files from the \LaTeX\ source.

\item \file{texjavahelp.lib}: a Java library that provides a \gls{gui} 
component to display the \gls{html} files created by \app{texjavahelpmk}.

\item \app{texjavahelpdemo}: a \gls{gui} demo application.
\end{itemize}

\chapter{The \stytext{texjavahelp} \LaTeX\ Package}
\label{sec:texjavahelpsty}

\pkgdef{texjavahelp}

\chapter{The \apptext{texjavahelpmk} Command Line Application}
\label{sec:texjavahelpmk}

\appdef{texjavahelpmk}

Use the \switch{help} switch 
(or \sswitch{help}) for help and the
\switch{version} switch
(or \sswitch{version}) for the version information.

\switchdef{help}
Display help message and exit.

\switchdef{version}
Display version information and exit.

\chapter{The \TeX\ Java Help Library (\filetext{texjavahelp.lib})}
\label{sec:texjavahelplib}

/\glsfmttext{app.texjavahelpmk}/

% Summary of texjavahelpmk switches
% Either:
%\listentry{texjavahelpmk}
% Or:
\listentrydescendents
 [title={Summary of \apptext{texjavahelpmk} Switches}]
 {app.texjavahelpmk}

\printmain

\printindex

\end{document}
