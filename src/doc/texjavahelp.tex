% arara: pdflatex
% arara: bib2gls: { group: on }
% arara: pdflatex
% arara: pdflatex
\documentclass[toc=listof]{scrreport}

\usepackage{texjavahelp}

\hypersetup{colorlinks,linkcolor=blue}

\title{\TeX\ Java Help}
\author{Nicola L.C. Talbot}

\GlsXtrLoadResources[\TeXJavaHelpGlsResourceOptions]

\begin{document}
\maketitle
\tableofcontents

\chapter{Introduction}
\label{sec:intro}

This is the documentation for \TeX\ Java Help.

\begin{warning}
Under construction.
\end{warning}

\begin{itemize}
\item \gls{texjavahelp}: a \LaTeX\ package to help create the \gls{pdf} version of
the documentation from the \LaTeX\ source.

\item \gls{texjavahelpmk}: a \gls{cli} application that uses the \TeX\ Parser
Library to create the \gls{html} files from the \LaTeX\ source.

\item \gls{texjavahelp.lib}: a Java library that provides a \gls{gui} 
component to display the \gls{html} files created by \gls{texjavahelpmk}.

\item \gls{texjavahelpdemo}: a \gls{gui} demo application.
\end{itemize}

\chapter{The \glsfmttext{texjavahelp} \LaTeX\ Package}
\label{sec:texjavahelpsty}

\stydef{texjavahelp}

\chapter{The \glsfmttext{texjavahelpmk} Command Line Application}
\label{sec:texjavahelpmk}

\clidef{texjavahelpmk}

Use the \gls{help} switch (or \shortswitch{help}) for help and the
\gls{version} switch (or \shortswitch{version}) for the version information.

\switchdef{help}
Display help message and exit.

\switchdef{version}
Display version information and exit.

\chapter{The \TeX\ Java Help Library (\glsfmttext{texjavahelp.lib})}

% Summary of texjavahelpmk switches
% Either:
%\listentry{texjavahelpmk}
% Or:
\listentrydescendents[title={Summary of \glsfmttext{texjavahelpmk} Switches}]{texjavahelpmk}

\printterms

\printindex

\end{document}
