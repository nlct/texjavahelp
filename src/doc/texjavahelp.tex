% arara: pdflatex
% arara: bib2gls: { group: on }
% arara: pdflatex
% arara: pdflatex
\documentclass[toc=listof]{scrreport}

\usepackage{texjavahelp}

\hypersetup{colorlinks,linkcolor=blue}

\newcommand{\TeXJavaHelp}{\texorpdfstring{\TeX}{TeX} Java Help}

\title{\TeXJavaHelp}
\author{Nicola L.C. Talbot\\\href{https://www.dickimaw-books.com/}{\nolinkurl{dickimaw-books.com}}}

\GlsXtrLoadResources[\TeXJavaHelpGlsResourceOptions]

\begin{document}
\maketitle
\tableofcontents

\chapter{Introduction}
\label{sec:intro}

I have a number of Java \gls{gui} applications, which require a manual.
Usually I create \gls{pdf} manuals using \pdfLaTeX\ or \LuaLaTeX, but
it's useful for a \gls{gui} to also have an in-application help, where 
an appropriate section can be opened for context-sensitive help (for example,
via a \widgetfmt{Help} button in a dialog).

In the past, I used \gls{JavaHelp} but it has a number of limitations and now
no longer seems to be supported. Ideally, I would like the same document source
to provide the offline \gls{pdf}, the \gls{html} and \gls{xml} files required
for the application's help system, and (if the documentation is particularly
large) \gls{html} files for the application's home page.

I've tried several ways to achieve this. The first was to use \LaTeX\ source
and convert to \gls{html}. The limitations of Swing's \varfmt{JEditorPane},
the complexity of adding support for \gls{JavaHelp}['s] custom
\xmltagfmt{object} format, and the need to create the supplementary \gls{xml}
files, meant that some additional work was required to get the correct output.

The second method I tried was to start with \gls{xml} source with a custom
script which generated both the \gls{html} and \gls{xml} files for \gls{JavaHelp}
and also the \LaTeX\ source for the \gls{pdf}. This proved the more reliable route
for the \gls{html} and \gls{xml} output, as it's much easier to convert from \gls{xml}
to \gls{html}, although I don't particularly like writing large documents in 
\gls{xml}.

One of the advantages of \gls{html} over \gls{pdf} is that it's more accessible.
This should mean that the in-application help ought to be more accessible than the
\gls{pdf} manual, but I couldn't find a way to allow the user to customise the
font without having to resort to editing the \gls{html} files (for example, to
make the font larger for visually impaired users or to change to a dyslexic font).

I'm now trying a different approach. Since most of my Java applications are in
some way related to \TeX, and I created the \gls{TeXParserLib} to parse
\LaTeX\ code to assist them, I decided to try using the \gls{TeXParserLib}
to parse the \LaTeX\ source code for the manual in order to create the 
\gls{html} and \gls{xml} files needed for a \gls{gui} help system. At the same
time, I decided to also create a library to provide a simple in-application help
system to replace \gls{JavaHelp}.

The biggest drawback is that the \gls{TeXParserLib} was originally designed to
perform limited parsing of documents: for extracting information, removing obsolete
or problematic code, or for converting short fragments into \gls{html}.
It wasn't intended for converting entire documents to \gls{html} and lacks support
for all but a handful of packages. The available support is essentially limited
to the support required for the various projects I've worked on. It's now 
principally geared towards \app{bib2gls} and \app{datatooltk}.

However, I have started using the \gls{TeXParserLib} to convert my \LaTeX\
package user manuals to \gls{html} to test the parsing capabilities. This means
that there's already support for my \sty{nlctuserguide} package. So I've started
by creating a new package, \sty{texjavahelp}, that's heavily based on
\sty{nlctuserguide} so that I can reuse much of the \gls{TeXParserLib}
support for \sty{nlctuserguide}. Unlike \sty{nlctuserguide}, which is 
designed for a self-contained source document, the \sty{texjavahelp} package
is designed to have accompanying \ext{bib} files for use with \app{bib2gls}.

The \TeX\ Java Help repository is at \url{https://github.com/nlct/texjavahelp}
and consists of:

\begin{itemize}
\item \sty{texjavahelp}: a \LaTeX\ package to help create the \gls+{pdf} version of
the documentation from the \LaTeX\ source. Requires the \sty{glossaries-extra}
package and \app{bib2gls}.

\item \app{texjavahelpmk}: a \gls{cli} application that uses the \TeX\ Parser
Library (with a custom implementation of the \sty{texjavahelp} package) 
to create the \gls+{html} and \gls+{xml} files (required by the 
\TeXJavaHelp\ library) from the \LaTeX\ source.

\item \file{texjavahelp.lib}: a Java library that provides a \gls{gui} 
component to display the \gls{html} files created by \app{texjavahelpmk}.

\item \app{texjavahelpdemo}: a \gls{gui} demo application to test 
\app{texjavahelpmk} and \file{texjavahelp.lib}.
\end{itemize}

\begin{information}
It's not necessary to use the \sty{texjavahelp} package and \app{texjavahelpmk}.
The \gls{html} and \gls{xml} files can be created manually in a text editor 
without them, but they must be in the correct format. The use of \app{bib2gls}
for the glossary and index provides information written to the \file{index.xml}
file that's used in the tooltip text when hovering the mouse over a hyperlink
in the help page.
\end{information}

\chapter{The \stytext{texjavahelp} \LaTeX\ Package}
\label{sec:texjavahelpsty}

\pkgdef{texjavahelp}

The \sty{texjavahelp} package requires the following packages:
\sty{glossaries-extra}, \sty{fontawesome}, \sty{twemojis},
\sty{upquote}, \sty{hologo}, \sty{xcolor}, \sty{tcolorbox}, \sty{hyperref},
and (optionally) \sty{tikz}. These are loaded automatically.
The \sty{glossaries-extra} package is loaded with the options:
\optval{record}{nameref} (which means that \app{bib2gls} should be used),
\opt{indexcounter}, \opt{floats}, \opt{nostyles}, 
\optvalm{stylemods}{tree,bookindex,topic} and \optval{style}{alttree}.

The following \sty{texjavahelp} package options are provided:
\optiondef{fontsymbols}
This option is the default, and defines \gls{tabsym} and \gls{upsym} in terms of
the font symbol commands \gls{barleftarrowrightarrowbar}
and \gls{baruparrow}. The \opt{fontsymbols} option will required a package that defines
these commands, such as \sty{stix} or \sty{boisk}, if those commands are required.

\optiondef{tikzsymbols}
This option automatically loads the \sty{tikz} package and 
defines \gls{tabsym} and \gls{upsym} as images.
The \opt{tikzsymbols} option counteracts the \opt{fontsymbols} option.

Another other options supplied to \sty{texjavahelp} will be passed to
the \sty{glossaries-extra} package.

The \sty{texjavahelp} package is intended to be used with \app{bib2gls}. This means
that the document preamble should contain:
\cmddef{GlsXtrLoadResources}
This command is provided by the \sty{glossaries-extra} package and is used 
in conjunction with \app{bib2gls}. It writes information
to the \ext{aux} file to be read by \app{bib2gls} and inputs
(if it exists) the \ext{glstex} file created by \app{bib2gls}.

For convenience, the \sty{texjavahelp} package provides:
\cmddef{TeXJavaHelpGlsResourceOptions}
This expands to the resource options customized for use with the
\sty{texjavahelp} package. This sets the \opt{entry-type-aliases}
and \opt{assign-fields} options to work with the \ext{bib} format described in
in \sectionref{sec:bibformat}.

Other resource options included in the definition of
\gls{TeXJavaHelpGlsResourceOptions}
are: \optvalm{selection}{recorded and deps and see}, \optvalm{category}{same as
original entry} and \opt{save-entry-count}. Additionally, this command
starts with:
\cmddef{TeXJavaHelpGlsFieldAdjustments}
which expands to the field adjustment resource options: 
\optval{description-case-change}{firstuc} and \optval{post-description-dot}{check}.
See the \app{bib2gls} user manual for further information on resource options.

\section{The \TeXJavaHelp\ \exttext{bib} Format}
\label{sec:bibformat}

The \ext{bib} format required by \app{bib2gls} is described in the
\app{bib2gls} user manual. The \gls{TeXJavaHelpGlsResourceOptions} command
sets up entry type aliases and field adjustments to assist formatting and work
with the semantic commands described in \sectionref{sec:semanticcmds}.
This means that custom entry types and fields are available that aren't
ordinarily recognised by \app{bib2gls}.

\section{Semantic Commands}
\label{sec:semanticcmds}

Semantic commands are commands that relate to a particular idea or meaning.
For example, \code{\gls{emph}\margm{text}} is a semantic command that indicates
emphasis. This may render \meta{text} in an italic font, but this isn't always the
case. In \gls{html}, this is equivalent to using
\code{\xmltagfmt{em}\meta{text}\xmltagfmt{/em}}.
The style may be changed, but the meaning remains.

As with other commands, semantic commands still have a defined syntax, but
there is no meaning associated with purely syntactic commands, such as
\gls{textit}. The commands \gls{emph} and \gls{textit} have the same syntax, in that
they both take a single argument, but \gls{emph} specifically means that the
argument should be emphasized, whereas \gls{textit} is a font changing
instruction.

The \sty{texjavahelp} package provides a number of semantic commands, not only
because it's good practice but also because it helps the \gls{TeXParserLib} to
produce more accessible \gls{html} with as little inline styling as possible.
The other reason is to allow the use of label prefixes to help disambiguate
closely related labels.  For example, the label \qtt{switch.help} is for the
command line switch \switch{help} whereas \qtt{menu.help} is for a \menu{help}
menu and \qtt{action.help} is for a \widget{help} button or some other type of
action widget.

\chapter{The \apptext{texjavahelpmk} Command Line Application}
\label{sec:texjavahelpmk}

\appdef{texjavahelpmk}

Use the \switch{help} switch 
(or \sswitch{help}) for help and the
\switch{version} switch
(or \sswitch{version}) for the version information.

\switchdef{help}
Display help message and exit.

\switchdef{version}
Display version information and exit.

\chapter{The \TeX\ Java Help Library (\filetext{texjavahelp.lib})}
\label{sec:texjavahelplib}

/\glsfmttext{app.texjavahelpmk}/

% Summary of texjavahelpmk switches
% Either:
%\listentry{texjavahelpmk}
% Or:
\listentrydescendents
 [title={Summary of \apptext{texjavahelpmk} Switches}]
 {app.texjavahelpmk}

\printmain

\printindex

\end{document}
